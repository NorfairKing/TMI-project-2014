
% \lhead{Tom Sydney Kerckhove \& Xavier Go\'as Aguililla}
% \rhead{\thepage /\pageref{LastPage}}

% \lfoot{Practicum}
\cfoot{\thepage\ van \pageref{LastPage} }
% \rfoot{Snijdende Cirkels}

\renewcommand{\headrulewidth}{0.4pt}
\renewcommand{\footrulewidth}{0.4pt}

% Index, x, y, r
\def\circle[#1,#2,#3,#4] {
  % The middlepoint
  \coordinate (C#1) at (#2,#3);
  
  % The point
  \draw[fill,color=black] (C#1) circle (1.5pt)
  node[left, yshift=-10pt, color=black] {$C_#1$};
  
  % The radius
  \draw[dotted] (C#1) -- ($(C#1) + (#4,0)$)
  node[left, yshift=-10pt, color=black] {$r_#1$};
  
  
  % The circle
  \draw[thick,color=black] (C#1) circle (#4);
  
}
  
\def\indexedcircle[#1,#2,#3,#4] {
  % The middlepoint
  \coordinate (C#1) at (#2,#3);
  
  % The point
  \draw[fill,color=black] (C#1) circle (1.5pt)
  node[left, yshift=-10pt, color=black] {$C_#1$};
  
  % The radius
  \draw[dotted] (C#1) -- ($(C#1) + (#4,0)$)
  node[left, yshift=-10pt, color=black] {$r_#1$};
  
  
  % The circle
  \draw[thick,color=black] (C#1) circle (#4);
  % Event points
  \coordinate (E#1_1) at (#2,#3-#4);
  \coordinate (E#1_2) at (#2,#3+#4);
  
  % \draw[fill, color=red] (E#1_1) circle (1.5pt) node[left, yshift=-10pt, color=red] {$e_{#1,1}$};
  % \draw[fill, color=red] (E#1_2) circle (1.5pt) node[left, yshift=-10pt, color=red] {$e_{#1,2}$};
  \draw[fill, color=red] (E#1_1) circle (1.5pt) node[left, yshift=-10pt, color=red] {};
  \draw[fill, color=red] (E#1_2) circle (1.5pt) node[left, yshift=-10pt, color=red] {};
}

\delimitershortfall-1sp
\newcommand\abs[1]{\left|#1\right|}

\theoremstyle{plain}
\newtheorem{inv}{Invariante}
\newtheorem{stl}{Stelling}[subsection]
\newtheorem{lemma}{Lemma}
\newtheorem*{gevolg}{Gevolg}

\theoremstyle{remark}
\newtheorem{geval}{Geval}