\section{Inleiding}
We bespreken drie verschillende algoritmen voor het vinden van de
snijpunten van een verzameling van $n$ cirkels $C$. De opgave van het
practicum herhalen is vrij zinloos, maar alvorens we overgaan tot de
werkelijke inhoud van dit verslag, willen we een aantal zaken
aankaarten met betrekking tot onze algoritmische notatie en de
implementatie van de algoritmen.

\paragraph{Notatie.} 
Wat betreft de notatie van de algoritmen volgen we in het algemeen de
wiskundige notatie voor verzamelingen en operaties daarop. De
algoritmes volgen globaal een imperatief stramien: de stappen worden
sequentieel uitgevoerd zoals bepaald door de besturingsstructuren in
het algoritme. Logische expressies en functies worden zoals hoort in
standaard logische notatie weergegeven. Dit staat enigszins in
contrast met onze implementatie.

\paragraph{Implementatie.} 
Wij hebben ervoor gekozen de algoritmen te
implementeren in Haskell, een functionele programmeertaal. Dit zorgt
ervoor dat de algoritmen op een andere manier moeten of kunnen worden
uitgedrukt dan in de klassieke notatie. Dit kan code veel leesbaarder
en beknopter maken. Anderzijds is het soms moeilijker om een
traditionele complexiteitsanalyse te maken: zo kent Haskell geen
iteratieve lussen, maar wordt overvloedig gebruik gemaakt van
recursie. We belichten sommige aspecten van onze implementatie
concreter in sectie \ref{sec:implementation}.

\paragraph{Invoer en uitvoer.} 
De invoer en uitvoer gebeurt via de standaard \textit{streams} voor
invoer en uitvoer op Linux. Het programma wordt dus als volgt
aangeroepen:
\begin{lstlisting}
Executable < input.txt > output.txt
\end{lstlisting}

\paragraph{Naamgeving.} 
We zullen de beschreven algoritmes aanduiden overeenkomstig hun
complexiteit. Zo noemen we de algoritmes respectievelijk `na\"ief',
`kwadratisch' en `linearitmisch'. In de tekst zelf zullen we vaak ook
refereren naar hun volgnummer in de lijst van algoritmen in de
inhoudstafel. `Algoritme 3' in de tekst komt dus ni\'et overeen met
het `algoritme 3' uit de opgave!

\paragraph{} 
Wij wensen u veel lees- en verbeterplezier.