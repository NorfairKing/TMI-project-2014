We bespreken drie verschillende algoritmen voor het vinden van de
snijpunten van een verzameling cirkels $C$. De opgave van het
practicum herhalen is vrij zinloos, maar alvorens we overgaan tot de
werkelijke inhoud van dit verslag, willen we een aantal zaken
aankaarten met betrekking tot onze algoritmische notatie en de
implementatie van de algoritmen.

Wat betreft de notatie van de algoritmen volgen we in het algemeen de
wiskundige notatie voor verzamelingen en operaties daarop. De
algoritmes volgen globaal een imperatief stramien: de stappen worden
sequentieel uitgevoerd zoals bepaald door de besturingsstructuren in
het algoritme. Logische expressies en functies worden zoals hoort in
standaard logische notatie weergegeven.