\section{Implementatie}
In deze sectie beschrijven we de implementatiedetails van de algoritmes en de gebruikte gegevensstructuren.
\todo{vermeld de problemen met haskell}

\subsection{Naief}
Het naieve algoritme maakt gebruik van staartrecursie zodat er nooit twee cirkels maar dan \'e\'en keer met elkaar vergeleken worden.

\subsection{Kwadratisch}
Het kwadratische algoritme maakt geen gebruik van externe datastructuren. Het gebruikt de ingebouwde lijsten van haskell om de status bij te houden.


\subsection{Linearitmisch}
het Linearitmische algoritme maakt gebruikt van een datastructuur genaamd `intervalmap' om de status bij te houden.
Op deze manier kunnen de cirkels met overlappende intervallen kunnen gezocht worden in $O(R\log N)$ tijd (waarbij $R$ het aantal cirkels met overlappende intervallen voorstelt).