\newpage
\section{Implementatie}
\label{sec:implementation}
In deze sectie beschrijven we de implementatiedetails van de
algoritmes en de gebruikte gegevensstructuren. We tonen eveneens
stukjes code. We garanderen niet dat deze code zal compileren wanneer
ze gekopieerd wordt. De code die hier staat is enigszins vereenvoudigd
om de leesbaarheid te vergoten.

\subsection{Haskell}
We hebben de code geschreven in Haskell. Haskell is een programmeertaal die gefundeerd is op de volgende principes:

\begin{enumerate}
\item Haskell is \textbf{lui}: een expressie zal enkel ge\"evalueerd
worden als het resultaat van die evaluatie nodig is.
\item Haskell is \textbf{puur}: functies in Haskell hebben geen
side-effects. Hierdoor is er een garantie dat voor een bepaalde input
een functie altijd dezelfde output zal teruggeven, en kunnen
programma's vrij eenvoudig geparallelliseerd worden.
\item Haskell is \textbf{functioneel}: functies in Haskell zijn in
tegenstelling tot talen als Java zgn. \textit{first-class citizens}:
we kunnen functies manipuleren als gewone data.
\end{enumerate}

Dit vereenvoudigde het implementeren van de wiskundige aspecten van
dit project en verkortte de code in het geheel. Bepaalde aspecten
zorgden evenwel voor een aantal problemen.

\subsubsection{Evaluatie en timing}

Haskell wordt niet per se sequentieel ge\"evalueerd. Code wordt
doorgaans enkel ge\"evalueerd wanneer het resultaat nodig is voor
uitvoer.  Een resultaat dat nooit opgevraagd wordt zal dus ook nooit
berekend worden. Voorts is het ook zo dat functies die niet aan I/O
doen in Haskell niet beschouwd worden als procedures, maar wel als
definities van functies. Wanneer we zo'n functie toepassen op haar
argumenten, wordt binnen het paradigma dat Haskell gebruikt eigenlijk
niets berekend!

We hebben het algoritme zo ge\"implementeerd dat de cirkels eerst
worden ingelezen, de snijpunten dan worden berekend en pas daarna de
snijpunten worden uitgelezen. Enkel het berekenen van de snijpunten
wordt dan getimed. De snijpunten worden dus niet meer op een luie
manier berekend. Het programma is hierdoor natuurlijk trager geworden.

Voor het timen van onze code gebruiken we het `Criterion' pakket op hackage.\footnote{\url{https://hackage.Haskell.org/package/criterion}}
Voor het garanderen van striktheid van de code gebruiken we het `DeepSeq' pakket op hackage.\footnote{\url{https://hackage.Haskell.org/package/deepseq}}

\subsubsection{Lijsten}
Operaties op de ingebouwde lijsten in Haskell hebben de volgende complexiteiten.
\begin{figure}[H]
\centering
\begin{tabular}{|c|c|}
\hline
Operatie & Complexiteit\\
\hline\hline
pattern matching & $O(1)$\\\hline
concat & $O(N+M)$\\\hline
map & $O(N)$\\\hline
nub (uniques) & $O(N\log(N))$\\\hline
sort & $O(N\log(N))$\\\hline
\end{tabular}
\label{fig:list_performance}
\caption{Operaties op lijsten}
\end{figure}

\newpage
\subsection{Naief}
Het naieve algoritme maakt gebruik van staartrecursie zodat er nooit twee cirkels maar dan \'e\'en keer met elkaar vergeleken worden. Dit algoritme werkt steeds in $O(N^2)$ tijd.

\begin{figure}[H]
\begin{lstlisting}[language=haskell]
intersections :: [Circle] -> [Position]
intersections [] = []
intersections [_] = []
intersections l = nub $ go l
    where
        go [] = []
        go [_] = []
        go (c:cs) = concatMap (circlesIntersections c) cs ++ go cs
\end{lstlisting}
\label{imp:naive}
\caption{Functionele code voor het na\"ieve algoritme}
\end{figure}


\subsection{Kwadratisch}
Het kwadratische algoritme maakt geen gebruik van externe datastructuren. Het gebruikt de ingebouwde lijsten van Haskell om de status bij te houden. Dit algoritme heeft een slechtste-geval complexiteit van $O(N^2)$ en een beste-gevalcomplexiteit van $O(N\log(N)$. De `events' moeten immers nog steeds gesorteerd worden;

\begin{figure}[H]
\begin{lstlisting}[language=haskell]
intersections :: [Circle] -> [Position]
intersections [] = []
intersections [c] = []
intersections cs = nub $ go (sort $ eventPointss cs) []
    where
        go :: [Event] -> [Circle] -> [Position]
        go [e] act = []
        go (Insert c : evl) act 
        	= concatMap (circlesIntersections c) act 
        	++ go evl (c:act)
        go (Delete c : evl) act = go evl (delete c act)
\end{lstlisting}
\label{imp:quadratic}
\caption{Functionele code voor het kwadratische algoritme}
\end{figure}

\newpage
\subsection{Linearitmisch}
Het Linearitmische algoritme maakt gebruikt van een datastructuur genaamd `intervalmap' om de status bij te houden.
Op deze manier kunnen de cirkels met overlappende intervallen kunnen gezocht worden in $O(R\log N)$ tijd (waarbij $R$ het aantal cirkels met overlappende intervallen voorstelt).

\begin{figure}[H]
\begin{lstlisting}[language=haskell]
intersections :: [Circle] -> [Position]
intersections []  = []
intersections [_] = []
intersections cs = nub $ go (sort' eventPointss cs) I.empty
    where
        go :: [Event] -> IntervalMap Position Circle -> [Position]
        go [e] _ = []
        go (Insert c : es) act = intersects ++ next
            where
                intersects 
                	= concatMap 
                	  (circlesIntersections c) 
                	  overlapping
                overlapping = intersecting act thisInterval
                
                next = go es newAct
                newAct = insert (interval c) c act
              
        go (Delete c : es) act = go es newAct
            where newAct = delete (circleInterval c) act
\end{lstlisting}
\label{imp:linearithmic}
\caption{Functionele code voor het linearitmische algoritme}
\end{figure}