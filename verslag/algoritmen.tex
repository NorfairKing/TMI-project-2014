
\subsection{Het snijpunt van twee cirkels berekenen}
\label{sec:snijpunt}

\subsubsection{Algoritme}
\begin{algorithm}[H]
  \SetAlgoLined
  \KwIn{twee cirkels, $c$ en $c'$ met resp middelpunten $p_{1}, p_{2}$ en stralen $r_{1}, r_{2}$}
  \KwOut{waar als en slechts als de twee cirkels snijden}
  \SetKwProg{Fn}{Procedure}{:}{end}
  \Fn{intersect($c$,$c'$)}{
    d $\leftarrow$ distance(p1,p2)\\
    \eIf{($d \leq r_1 + r_2) \land (d \geq abs (r_1 - r_2))$}{
      \Return{true}
    }{
      \Return{false}
    }
  }

  \caption{Nagaan of twee cirkels snijden}
\end{algorithm}

\begin{algorithm}[H]
  \SetAlgoLined
  \KwIn{twee cirkels, $c$ en $c'$ met resp. middelpunten $p_{1} = (x_1,y_1), p_{2} = (x_2,y_2)$ en stralen $r_{1}, r_{2}$}
  \KwOut{geen, of twee snijpunten van de twee cirkels (die dan identiek zijn)}
  \SetKwProg{Fn}{Procedure}{:}{end}
  \Fn{intersections($c$,$c'$)}{
    \eIf{$intersect(c1, c2) \land c1 \neq c2$}{
      $d \leftarrow distance(p_1, p_2)$\\
      $\alpha \leftarrow \frac{r_1^2 -r_2^2}{2d^2}$\\
      $s \leftarrow \frac{x_1+x_2}{2} + \alpha(x_2-x_1)$\\
      $t \leftarrow \frac{y_1+y_2}{2} + \alpha(y_2-y_1)$\\

      $\delta \leftarrow \frac{1}{4}  \sqrt { (d+r_1+r_2)(d+r_1-r_2)(d-r_1+r_2)(r_1+r_2-d)}$\\
      $x_1' \leftarrow s + 2\delta\frac{y_1-y_2}{d^2}$\\
      $x_2' \leftarrow s - 2\delta\frac{y_1-y_2}{d^2}$\\
      $y_1' \leftarrow t - 2\delta\frac{x_1-x_2}{d^2}$\\
      $y_2' \leftarrow t + 2\delta\frac{x_1-x_2}{d^2}$\\
      \Return{$\left\{(x_1',y_1'),(x_2',y_2')\right\}$}
    }{
      \Return{$\varnothing$}
    }
  }
  \caption{Snijpunten van twee cirkels berekenen}
\end{algorithm}

\subsubsection{Correctheidsbewijs}
\subsubsection{Complexiteit}


\subsection{Na\"ief}
\label{sec:naief}

\subsubsection{Algoritme}

\begin{algorithm}[H]
  \SetAlgoLined
  \KwIn{een lijst van $n$ cirkels $C$, gegeven door hun middelpunt en straal}
  \KwOut{een verzameling van $S$ snijpunten $R$ van de cirkels in $C$}
  \SetKwProg{Fn}{Procedure}{:}{end}
  \Fn{intersections1($C$)}{
    R $\leftarrow \varnothing$\;
    \For{$i\leftarrow 1$ \KwTo $n$}{
      \For{$j\leftarrow i$ \KwTo $n$}{
        $R \leftarrow R \cup intersections(C(i),C(j))$
      }

    }
    \Return{R}
  }
  \caption{Na\"ieve aanpak (imperatief)}
  \label{algo:naive}
\end{algorithm}

\subsubsection{Correctheidsbewijs}
\subsubsection{Complexiteit}
De naïeve aanpak heeft complexiteit $O(n^2)$.

% \begin{figure}
%   \[
%   intersections\ cs\ = \{\ circleIntersections\ c1\ c2\ |\ c1 \leftarrow\ cs,\ c2\ \leftarrow\\ cs \}
%   \]
%   \label{naief_functioneel}
%   \caption{Na\"ieve aanpak (functioneel)}
% \end{figure}


\subsection{Kwadratisch}
\label{sec:kwadratisch}

\subsubsection{Algoritme}

\begin{algorithm}[H]
  \SetAlgoLined
  \KwIn{een lijst van $n$ cirkels $C$, gegeven door hun middelpunt en straal}
  \KwOut{een verzameling van $S$ snijpunten $R$ van de cirkels in $C$}
  \SetKwProg{Fn}{Procedure}{:}{end}
  \Fn{intersections2($C$)}{
    $E \leftarrow \varnothing$\;
    \For{$c \in C$}{
      $E \leftarrow E \cup events(c)$;
    }
    $E \leftarrow sort(E)$\;
    $T, R \leftarrow \varnothing$\;
    \For{$e \in E$}{
      \uIf {e == insert c} {
        \For{$c' \in T$} {
          $R \leftarrow R \cup intersections(c,c')$
        }
        $T \leftarrow T \cup \left\{c\right\}$
      } \ElseIf {e == delete c} {
        $T \leftarrow T \setminus \left\{c\right\} $
      }
    }
    \Return{R}
  }
  \caption{Kwadratische aanpak (imperatief)}
\end{algorithm}

\begin{algorithm}[H]
  \SetAlgoLined
  \KwIn{een cirkel $c$ met middelpunt $(x, y)$ en straal $r$}
  \KwOut{twee \textit{events} $e_1$ en $e_2$,
    corresponderend aan het toevoegen en verwijderen van $c$ aan de
    doorlooplijn, waarbij elk event ook de $y$-co\"ordinaat waarop
    het gebeurt met zich meedraagt}
  \SetKwProg{Fn}{Procedure}{:}{end}
  \Fn{events($c$)}{
    $y_1 \leftarrow  y - r$\;
    $y_2 \leftarrow  y + r$\;
    $R \leftarrow \left\{(y_1, insert\ c), (y_2, delete\ c)\right\}$\;
    \Return{R}
  }
  \caption{Events}
\end{algorithm}

% \begin{figure}
%   \[
%   intersections\ cs = go\ (sort\ (events\ cs))\ [\ ]
%   \]
%   \[
%   where 
%   \]
%   \[
%   \begin{array}{l c l}
%     go\ [\ ] &= &[\ ]\\
%     go (Insert\ c\ :\ es) T &= &(map\ (circleIntersections\ c)\ T) \cup (go\ es\ T\cup\{c\})\\
%     go (Delete\ c\ :\ es) T &= &go\ es\ T\backslash\{c\}\\
%   \end{array}
%   \]
%   \label{naief_functioneel}
%   \caption{Kwadratische aanpak (functioneel)}
% \end{figure}
\subsubsection{Correctheidsbewijs}
\subsubsection{Complexiteit}


\subsection{Linearitmisch}
\label{sec:linearitmisch}

\subsubsection{Algoritme}

\begin{algorithm}[H]
  \SetAlgoLined
  \KwIn{een cirkel $c$ met middelpunt $(x, y)$ en straal $r$}
  \KwOut{een interval $I$ corresponderend aan het interval tussen de uiterste $x$-co\"ordinaten van $c$}
  \SetKwProg{Fn}{Procedure}{:}{end}
  \Fn{interval($c$)}{
    $x_1 \leftarrow x - r$\;
    $x_2 \leftarrow x + r$\;
    $I \leftarrow [x_1, x_2]$\;
    \Return{I}
  }
  \caption{Breedte-interval van een cirkel}
\end{algorithm}
\label{algo:interval}

\begin{algorithm}[H]
  \KwIn{een lijst van $n$ cirkels $C$, gegeven door hun middelpunt en straal}
  \KwOut{een verzameling van $S$ snijpunten $R$ van de cirkels in $C$}
  \SetAlgoLined
  \SetKwProg{Fn}{Procedure}{:}{end}
  \Fn{intersections3($C$)}{
    $E \leftarrow \varnothing$\;
    \For{$c \in C$}{
      $E \leftarrow E \cup events(c)$;
    }
    $E \leftarrow sort(E)$\;
    $T, R \leftarrow \varnothing$\;
    \For{$e \in E$}{
      \uIf {e == insert c} {
        \For{$c' \in T$} {
          \If {$interval(c) \cap interval(c') \neq \varnothing$} {
            $R \leftarrow R \cup intersections(c,c')$
          }
        }
        $T \leftarrow T \cup \left\{c\right\}$
      } \ElseIf {e == delete c} {
        $T \leftarrow T \setminus \left\{c\right\} $
      }
    }
    \Return{R}
  }
  \caption{Linearitmische aanpak (imperatief)}
\end{algorithm}
\subsubsection{Correctheidsbewijs}
\subsubsection{Complexiteit}
