
\subsection{Snijpunt van twee cirkels berekenen}
\label{sec:snijpunt}


\begin{figure}
\begin{algorithm}[H]
  \KwData{twee cirkels, $c$ en $c'$ met resp middelpunten $p_{1}, p_{2}$ en stralen $r_{1}, r_{2}$}
  \KwResult{waar als en slechts als de twee cirkels snijden}
  $d \leftarrow \Vert p_2-p_1\Vert$\\
  \Return{$d \leq r_1 + r_2 \land d \geq abs (r_1 - r_2)$}
 \caption{Nagaan of twee cirkels snijden}
\end{algorithm}
\label{algo:snijden}
\end{figure}

\begin{figure}
\begin{algorithm}[H]
  \KwData{twee cirkels, $c$ en $c'$ met resp. middelpunten $p_{1} = (x_1,y_1), p_{2} = (x_2,y_2)$ en stralen $r_{1}, r_{2}$}
  \KwResult{geen, of twee snijpunten van de twee cirkels (die dan identiek zijn)}
  \eIf{$snijden(c1, c2) \land c1 \neq c2$}{
    $d \leftarrow afstand(p_1, p_2)$\\
    $\alpha \leftarrow \frac{r_1^2 -r_2^2}{2d^2}$\\
    $s \leftarrow \frac{x_1+x_2}{2} + \alpha(x_2-x_1)$\\
    $t \leftarrow \frac{y_1+y_2}{2} + \alpha(y_2-y_1)$\\

    $\delta \leftarrow \frac{1}{4}  \sqrt { (d+r_1+r_2)(d+r_1-r_2)(d-r_1+r_2)(r_1+r_2-d)}$\\
    $x_1' \leftarrow s + 2\delta\frac{y_1-y_2}{d^2}$\\
    $x_2' \leftarrow s - 2\delta\frac{y_1-y_2}{d^2}$\\
    $y_1' \leftarrow t - 2\delta\frac{x_1-x_2}{d^2}$\\
    $y_2' \leftarrow t + 2\delta\frac{x_1-x_2}{d^2}$\\
    \Return{$\left\{(x_1',y_1'),(x_2',y_2')\right\}$}
  }{
    \Return{$\varnothing$}
  }

 \caption{Snijpunten van twee cirkels berekenen}
\end{algorithm}
  \label{algo:snijpunten}
\end{figure}


\subsection{Na\"ief}
\label{sec:naief}

De naïeve aanpak heeft complexiteit $O(n^2)$ en vinden we beschreven in
figuur \ref{algo:naive}. 

\begin{figure}
  \begin{algorithm}[H]
    \KwData{een lijst van $n$ cirkels $C$, gegeven door hun middelpunt en straal}
    \KwResult{een verzameling van $S$ snijpunten $R$ van de cirkels in $C$}
    R $\leftarrow \varnothing$\;
    \For{$i\leftarrow 1$ \KwTo $n$}{
      \For{$j\leftarrow i$ \KwTo $n$}{
        $R \leftarrow R \cup snijpunten(C(i),C(j))$
      }

    }
    \Return{R}
    \caption{Na\"ieve aanpak (imperatief)}
  \end{algorithm}
  \label{algo:naive}
\end{figure}

\begin{figure}
\[
intersections\ cs\ = \{\ circleIntersections\ c1\ c2\ |\ c1 \leftarrow\ cs,\ c2\ \leftarrow\\ cs \}
\]
\label{naief_functioneel}
\caption{Na\"ieve aanpak (functioneel)}
\end{figure}


\subsection{Kwadratisch}
\label{sec:kwadratisch}

\begin{figure}
  \begin{algorithm}[H]
    \KwData{een lijst van $n$ cirkels $C$, gegeven door hun middelpunt en straal}
    \KwResult{een verzameling van $S$ snijpunten $R$ van de cirkels in $C$}
    $E \leftarrow \varnothing$\;
    \For{$c \in C$}{
      $E \leftarrow E \cup events(c)$;
    }
    $E \leftarrow sorteer(E)$\;
    $T, R \leftarrow \varnothing$\;
    \For{$e \in E$}{
      \uIf {e == insert c} {
        \For{$c' \in T$} {
          $R \leftarrow R \cup snijpunten(c,c')$
        }
        $T \leftarrow T \cup \left\{c\right\}$
      } \ElseIf {e == delete c} {
        $T \leftarrow T \setminus \left\{c\right\} $
      }
    }
    \Return{R}
    \caption{Kwadratische aanpak (imperatief)}
  \end{algorithm}
\end{figure}

\begin{figure}
  \begin{algorithm}[H]
    \KwData{een cirkel $c$ met middelpunt $(x, y)$ en straal $r$}
    \KwResult{twee \textit{events} $e_1$ en $e_2$,
      corresponderend aan het toevoegen en verwijderen van $c$ aan de
      doorlooplijn, waarbij elk event ook de $y$-co\"ordinaat waarop
      het gebeurt met zich meedraagt}
    $x_1 \leftarrow  y - r$\;
    $x_2 \leftarrow  y + r$\;
    $R \leftarrow \left\{(y_1, insert\ c), (y_2, delete\ c)\right\}$\;
    \Return{R}
    \caption{Events}
  \end{algorithm}
\end{figure}

\begin{figure}
\[
intersections\ cs = go\ (sort\ (events\ cs))\ [\ ]
\]
\[
where 
\]
\[
\begin{array}{l c l}
go\ [\ ] &= &[\ ]\\
go (Insert\ c\ :\ es) T &= &(map\ (circleIntersections\ c)\ T) \cup (go\ es\ T\cup\{c\})\\
go (Delete\ c\ :\ es) T &= &go\ es\ T\backslash\{c\}\\
\end{array}
\]
\label{naief_functioneel}
\caption{Kwadratische aanpak (functioneel)}
\end{figure}

\begin{figure}
  \begin{algorithm}[H]
    \KwData{een cirkel $c$ met middelpunt $(x, y)$ en straal $r$}
    \KwResult{een interval $I$ corresponderend aan het interval tussen de uiterste $x$-co\"ordinaten van $c$}
    $x_1 \leftarrow x - r$\;
    $x_2 \leftarrow x + r$\;
    $I \leftarrow [x_1, x_2]$\;
    \Return{R}
    \caption{Breedte-interval van een cirkel}
  \end{algorithm}
  \label{algo:interval}
\end{figure}

\subsection{Linearitmisch}
\label{sec:linearitmisch}
\begin{figure}
  \begin{algorithm}[H]
    \KwData{een lijst van $n$ cirkels $C$, gegeven door hun middelpunt en straal}
    \KwResult{een verzameling van $S$ snijpunten $R$ van de cirkels in $C$}
    $E \leftarrow \varnothing$\;
    \For{$c \in C$}{
      $E \leftarrow E \cup events(c)$;
    }
    $E \leftarrow sorteer(E)$\;
    $T, R \leftarrow \varnothing$\;
    \For{$e \in E$}{
      \uIf {e = insert c} {
        \For{$c' \in T$} {
          \If {$interval(c) \cap interval(c') \neq \varnothing$} {
            $R \leftarrow R \cup snijpunten(c,c')$
          }
        }
        $T \leftarrow T \cup {c}$
      } \ElseIf {e = delete c} {
        $T \leftarrow T \setminus \left\{c\right\} $
      }
    }
    \Return{R}
    \caption{Linearitmische aanpak (imperatief)}
  \end{algorithm}
\end{figure}
