\newpage
\section{Besluit}
Het is mogelijk om alle snijpunten te vinden van een verzameling cirkels in $O(N+S)\log(N)$ tijd. Dit is echter niet steeds  wenselijk. Indien we op voorhand kunnen weten of er eerder veel of eerder weinig snijpunten zullen zijn, kunnen we het meest gepaste algoritme kiezen.
Algoritme drie is duidelijk beter wanneer er weinig snijpunten zijn. Algoritme \'e\'en en twee zijn duidelijk beter wanneer er veen snijpunten zijn.
In het algemeen lijkt het het best om voor algoritme twee te kiezen.


\section{Reflectie}

\subsection{Haskell sequence}
In heel het programma gebruiken we de ingebouwde lijsten van Haskell. Het datatype `Sequence' heeft echter, in de meeste gevallen en voor de meeste operaties, een betere slechtste geval-complexiteit. De totale uitvoeringstijd voor grote problemen zouden we zo nog kunnen inkorten.

\subsection{Compilatie optimalisaties}
De GHC' compiler die we gebruikt hebben biedt nog optimalisaties aan die we niet gebruikt hebben. Zo hebben we geprobeerd om de `llvm' vlag aan te zetten, maar omdat we daar een ongeteste versie van gebruikten resulteerde dit in compilatie fouten.