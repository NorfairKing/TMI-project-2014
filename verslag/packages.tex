\usepackage[dutch]{babel}
\usepackage{listings}

% Voor algoritmes
\usepackage{algorithm2e}

% Voor todo's
\usepackage{todonotes}

% Voor wiskunde
\usepackage{amsmath}
\usepackage{amsfonts}
\usepackage{amssymb}
\usepackage{amsthm}

% font
\usepackage{libertine}
\usepackage[libertine]{newtxmath}

% Om het totaal aantal pagina's te tellen
\usepackage{lastpage}
\usepackage{afterpage}

% Voor tekeningen
\usepackage{tikz}
\usetikzlibrary{decorations}
\usetikzlibrary{calc}

% Nog tekeningen
\usepackage{pgfplots}

% SVG tekeningen
\usepackage{svg}

% Om figuren op de juiste plaats te krijgen
\usepackage{float}

% Om de marges aan te passen
\usepackage[left=2cm,right=2cm,top=2cm,bottom=2cm]{geometry}

% Voor headers en footers
\usepackage{fancyhdr}
