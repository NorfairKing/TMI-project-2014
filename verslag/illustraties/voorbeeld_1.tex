\begin{figure}[H]
  \centering
  % \resizebox {\textwidth} {!} {
    \begin{tikzpicture}[scale=1]
      % grid
      \draw[very thin,color=lightgray] (0,0) grid (7,7);
      \draw [<->,thick] (0,7) node (yaxis) [left] {Y} |- (7,0) node (xaxis) [right] {X};
      
      \circle[1,2,1.5,1]
      \circle[2,4.5,2,1]    
      \circle[3,5.5,3,1]
      \circle[4,5,5.5,1]
      

      \draw[thick,color=blue,arrows={Triangle[scale=1]-Triangle[scale=1]}] (C1) -- (C2);
      \draw[thick,color=blue,arrows={Triangle[scale=1]-Triangle[scale=1]}] (C1) -- (C3);
      \draw[thick,color=blue,arrows={Triangle[scale=1]-Triangle[scale=1]}] (C1) -- (C4);
      \draw[thick,color=blue,arrows={Triangle[scale=1]-Triangle[scale=1]}] (C2) -- (C3);
      \draw[thick,color=blue,arrows={Triangle[scale=1]-Triangle[scale=1]}] (C2) -- (C4);
      \draw[thick,color=blue,arrows={Triangle[scale=1]-Triangle[scale=1]}] (C3) -- (C4);

      
    \end{tikzpicture}
  % }
  \label{fig:voorbeeld_1}
  \caption{Nagekeken cirkels bij algoritme 1}
\end{figure}